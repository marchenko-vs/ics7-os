\documentclass{bmstu}

\begin{document}

\makereporttitle
    {Информатика и системы управления}
    {Программное обеспечение ЭВМ и информационные технологии}
    {лабораторной работе №~1 (часть 2)}
    {Операционные системы}
    {Обработчик прерывания от системного таймера в Windows и Unix}
    {}
    {ИУ7-53Б}
    {В.~Марченко}
    {Н.~Ю.~Рязанова}
    {}

\maketableofcontents

\chapter{Функции обработчика прерывания от системного таймера}

Обработчик прерывания от системного таймера в системе имеет наивысший приоритет. 
По этой причине он не может быть прерван ни одним другим процессом. 
Следовательно, обработчик прерывания от системного таймера должен завершаться как можно быстрее.

\section{Unix}
    
По тику:
\begin{enumerate}
\item[1)] инкремент счетчика реального времени;
\item[2)] обновление статистики использования процессора текущим процессом (инкремент поля \textbf{p\_cpu} структуры \textbf{proc});
\item[3)] декремент кванта текущего потока;
\item[4)] декремент счетчиков времени до отправления на выполнение отложенных вызовов (при достижении счетчиком нуля происходит выставление флага для обработчика отложенного вызова).
\end{enumerate}

По главному тику.
\begin{enumerate}
\item Пробуждение системных процессов (<<пробуждение>> --- регистрация отложенного вызова процедуры \textbf{wakeup}, которая перемещает дескрипторы процессов из списка <<спящие>> в очередь <<готовы к выполнению>>).
\item Регистрация отложенныx вызовов функций, которые относятся к работе планировщика (в системе System V Release 4.0 можно зарегистрировать отложенный вызов с помощью \textbf{timeout(void (*fn)(), caddr\_t arg, long delta)}, где \textbf{fn()} --- функция, которую необходимо запустить, \textbf{arg} --- аргументы, которые получит \textbf{fn()}, \textbf{delta} --- временной интервал (в тиках процессора), через который \textbf{fn()} должна быть вызвана).
\item Декремент счетчика времени, которое осталось до отправления одного из следующих сигналов:
\begin{enumerate}
\item[1)] \textbf{SIGVTALRM} --- сигнал, посылаемый процессу по истечении времени, заданного в <<виртуальном>> таймере;
\item[2)] \textbf{SIGPROF} --- сигнал, посылаемый процессу по истечении времени, заданного в таймере профилирования;
\item[3)] \textbf{SIGALRM} --- сигнал, посылаемый процессу по истечении времени, предварительно заданного функцией \textbf{alarm()}.	
\end{enumerate}
\end{enumerate}

По кванту:
\begin{enumerate}
\item[1)] отправка текущему процессу сигнала \textbf{SIGXCPU}, если он израсходовал выделенный ему квант времени (при получении сигнала обработчик сигнала прерывает выполнение процесса).
\end{enumerate}

\section{Windows}

По тику:
\begin{enumerate}
\item[1)] инкремент счетчика системного времени;
\item[2)] декремент счетчиков времени отложенных задач;
\item[3)] декремент кванта текущего потока (декремент происходит на величину, равную количеству тактов процессора, произошедших за тик);
\item[4)] если активен механизм профилирования ядра, то инициализация отложенного вызова обработчика ловушки профилирования ядра с помощью постановки объекта в очередь \textbf{DPC} (обработчик ловушки профилирования регистрирует адрес команды, выполнявшейся на момент прерывания).
\end{enumerate}

По главному тику:
\begin{enumerate}
\item[1)] освобождение объекта <<событие>>, которое ожидает диспетчер настройки баланса (диспетчер настройки баланса по событию от таймера сканирует очередь готовых процессов и повышает приоритет процессов, которые находились в состоянии ожидания дольше 4-х секунд). 
\end{enumerate}

По кванту:
\begin{enumerate}
\item[1)] инициация диспетчеризации потоков (добавление соответствующего объекта в очередь \textbf{DPC}).
\end{enumerate}

\chapter{Пересчет динамических приоритетов}

В операционных системах семейств Unix и Windows только приоритеты пользовательских процессов могут динамически пересчитываться.

\section{Unix}

Очередь процессов, готовых к выполнению, формируется согласно приоритетам и принципу вытесняющего циклического планирования, то есть сначала выполняются процессы с большим приоритетом, а процессы с одинаковым приоритетом выполняются по алгоритму RR (round-robin). 
В случае, если процесс с более высоким приоритетом поступает в очередь процессов, готовых к выполнению, планировщик вытесняет текущий процесс и предоставляет ресурс более приоритетному
процессу. 
Приоритет процесса задается любым целым числом, которое находится в диапазоне от 0 до 127 (чем меньше это число, тем выше приоритет). 
Приоритеты от 0 до 49 зарезервированы для ядра. Следовательно, прикладные процессы могут обладать приоритетом в диапазоне от 50 до 127.

Структура \textbf{proc} содержит следующие поля, относящиеся к приоритетам:
\begin{enumerate}
\item[1)] \textbf{p\_pri} --- текущий приоритет планирования;
\item[2)] \textbf{p\_usrpri} --- приоритет режима задачи;
\item[3)] \textbf{p\_cpu} --- результат последнего измерения использования процессора;
\item[4)] \textbf{p\_nice} --- фактор <<любезности>>, который устанавливается пользователем.
\end{enumerate}

Поля \textbf{p\_pri} и \textbf{p\_usrpri} применяются для различных целей. 
Планировщик использует \textbf{p\_pri} для принятия решения о том, какой процесс направить на выполнение. 
Когда процесс находится в режиме задачи, значение его \textbf{p\_pri} идентично \textbf{p\_usrpri}. 
Когда процесс просыпается после блокирования в системном вызове, его приоритет будет временно повышен для того, чтобы дать ему предпочтение для выполнения в режиме ядра. 
Следовательно, планировщик использует \textbf{p\_usrpri} для хранения приоритета, который будет назначен процессу при возврате в режим задачи, а \textbf{p\_pri} --- для хранения временного приоритета для выполнения в режиме ядра. 
Процессу, ожидающему недоступный в данный момент ресурс, система определяет значение приоритета сна, выбираемое ядром из диапазона системных приоритетов и связанное с событием, вызвавшим это состояние. 
В таблице~\ref{tab:bsd} приведены значения приоритетов сна для систем 4.3 BSD UNIX и SCO UNIX (SCO OpenServer 5.0). 
Направление роста значений приоритета для этих систем различно --- в BSD UNIX большему значению соответствует более низкий приоритет. 
Когда процесс завершил выполнение системного вызова и находится в состоянии возврата в режим задачи, его приоритет сбрасывается обратно в значение текущего приоритета в режиме задачи. 
Измененный таким образом приоритет может оказаться ниже, чем приоритет какого-либо иного запущенного процесса. 
В этом случае ядро системы выполнит переключение контекста.

\begin{table}[h]
\caption{Приоритеты сна в операционной системе 4.3 BSD UNIX}
\label{tab:bsd}
\begin{center}
\begin{tabular}{ | l | l | l |  }
\hline
			\textbf{Событие} & \textbf{Приоритет} & \textbf{Приоритет} \\ 
            & \textbf{4.3 BSD UNIX} & \textbf{SCO UNIX} \\ \hline
			Ожидание загрузки в память & 0 & 95 \\
            сегмента/страницы & & \\ \hline
			Ожидание индексного дексриптора & 10 & 88 \\ \hline
			Ожидание ввода-вывода & 20 & 81 \\ \hline
			Ожидание буфера & 30 & 80 \\ \hline
			Ожидание терминального ввода  & & 75 \\ \hline
			Ожидание терминального вывода  & & 74 \\ \hline
			Ожидание завершения выполнения & & 73 \\ \hline
			Ожидание события --- & 40 & 66 \\
            низкоприоритетное состояние сна & & \\ \hline
\end{tabular}
\end{center}
\end{table}

Приоритеты в режиме задачи зависят от двух факторов.
\begin{enumerate}
\item От <<любезности>>. 
<<Любезность>> --- это целое число в диапазоне от 0 до 39 со значением 20 по умолчанию. 
Увеличение значения приводит к уменьшению приоритета. 
Пользователи могут повлиять на приоритет процесса при помощи изменения значений этого фактора, но только суперпользователь может увеличить приоритет процесса. 
Фоновые процессы автоматически имеют более высокие значения этого фактора.
\item От последней измеренной величины использования процессора.
\end{enumerate}

Системы разделения времени пытаются выделить процессорное время таким образом, чтобы конкурирующие процессы получили его примерно в равных количествах. 
Такой подход требует слежения за использованием процессора каждым из процессов. 
Каждую секунду ядро системы инициализирует отложенный вызов процедуры \textbf{schedcpu()}, которая уменьшает значение \textbf{p\_pri} каждого процесса исходя из фактора <<полураспада>>. В системе 4.3 BSD UNIX считается по формуле:
\begin{equation}
decay = \frac{2 \cdot load\_average}{2 \cdot load\_average + 1},
\end{equation}
где $load\_average$ --- это среднее количество процессов, находящихся в состоянии готовности к выполнению, за последнюю секунду.

\begin{equation}
p\_usrpri = \mbox{PUSER} + \frac{p\_cpu}{2} + 2 \cdot p\_nice,
\end{equation}
где $\mbox{PUSER}$ --- базовый приоритет в режиме задачи, равный 50.

Таким образом, если процесс в последний раз использовал большое количество процессорного времени, то его \textbf{р\_срu} будет увеличен. 
Это приведет к росту значения \textbf{p\_usrpri}, из чего следует понижение приоритета. 
Чем дольше процесс простаивает в очереди на выполнение, тем больше фактор <<полураспада>> уменьшает его \textbf{р\_срu}, что приводит к повышению его приоритета. 
Такая схема предотвращает бесконечное откладывание низкоприоритетных процессов. 
Применение данной схемы предпочтительно процессам, осуществляющим много операций ввода-вывода, в противоположность процессам, производящим много вычислений. 
То есть динамический пересчет приоритетов процессов в режиме задачи позволяет избежать бесконечного откладывания.

\section{Windows}

В Windows процессу при создании назначается приоритет. 
Относительно базового приоритета процесса потоку назначается относительный приоритет. 
Планирование осуществляется только на основании приоритетов потоков, готовых к выполнению: если поток с более высоким приоритетом становится готовым к выполнению, поток с более низким приоритетом вытесняется планировщиком. 
По истечению кванта времени текущего потока, ресурс передается самому приоритетному потоку в очереди готовых к выполнению потоков.

В Windows используется 32 уровня приоритета (от 0 до 31):
\begin{enumerate}
\item[1)] от 0 до 15 --- 16 изменяющихся уровней, из которых уровень 0 зарезервирован для потока обнуления страниц;
\item[2)] от 16 до 31 --- 16 уровней реального времени.
\end{enumerate}

Уровни приоритета потоков назначаются исходя из двух разных позиций: одной от Windows API и другой от ядра Windows. 
Сначала Windows API систематизирует процессы по классу приоритета, который им присваивается при создании:
\begin{enumerate}
\item[1)] реального времени (Real time) (4);
\item[2)] высокий (High) (3);
\item[3)] выше обычного (Above Normal) (6);
\item[4)] обычный (Normal) (2);
\item[5)] ниже обычного (Below Normal) (5);
\item[6)] уровень простоя (Idle) (1).
\end{enumerate}

Затем назначается относительный приоритет отдельных потоков внутри этих процессов. 
Здесь номера представляют изменение приоритета, применяющееся к базовому приоритету процесса:
\begin{enumerate}
\item[1)] критичный по времени (Time critical) (15);
\item[2)] наивысший (Highest) (2);
\item[3)] выше обычного (Above normal) (1);
\item[4)] обычный (Normal) (0);
\item[5)] ниже обычного (Below normal) (–1);
\item[6)] самый низкий (Lowest) (–2);
\item[7)] уровень простоя (Idle) (–15).
\end{enumerate}

Исходный базовый приоритет потока наследуется от базового приоритета процесса. 
Процесс по умолчанию наследует свой базовый приоритет у того процесса, который его создал. 
Соответствие между приоритетами Windows API и ядра системы приведено в таблице~\ref{tab:windowsapi}.

\begin{table}[h]
\caption{Соответствие между приоритетами Windows API и ядра Windows}
\label{tab:windowsapi}
\begin{center}
\begin{tabular}{ | l | l | l | l | l | l | l |  }
\hline
Приоритет & Real & High & Above & Normal & Below & Idle \\
& time & & normal & & normal & \\ \hline
Time critical & 31 & 15 & 15 & 15 & 15 & 15 \\ \hline
Highest & 26 & 15 & 12 & 10 & 8 & 6 \\ \hline
Above normal & 25 & 14 & 11 & 9 & 7 & 5 \\ \hline
Normal & 24 & 13 & 10 & 8 & 6 & 4 \\ \hline
Below normal & 23 & 12 & 9 & 7 & 5 & 3 \\ \hline
Lowest  & 22 & 11 & 8 & 6 & 4 & 2 \\ \hline
Idle  & 16 & 1 & 1 & 1 & 1 & 1 \\ \hline
\end{tabular}
\end{center}
\end{table}

В Windows есть диспетчер настройки баланса, который активизируется каждую секунду для возможной инициации событий, связанных с планированием и управлением памятью. 
Если обнаружены потоки, ожидающие выполнения более 4-х секунд, диспетчер настройки баланса повышает их приоритет до 15. 
Когда истекает квант, приоритет потока снижается до базового приоритета. 
Если поток не был завершен за квант времени или был вытеснен потоком с более высоким приоритетом, то после снижения приоритета поток возвращается в очередь готовых потоков.

Для того, чтобы минимизировать расход процессорного времени, диспетчер настройки баланса сканирует только 16 потоков и повышает приоритет не более чем у 10 потоков за одни проход. 
Когда обнаружится 10 потоков, приоритет которых следует повысить, диспетчер настройки баланса прекращает сканирование. 
При следующем проходе он возобновляет сканирование с того места, где сканирование было прервано.

Планировщик может повысить текущий приоритет потока в динамическом диапазоне (от 1 до 15) ввиду следующих причин:
\begin{enumerate}
\item[1)] повышение вследствие событий планировщика или диспетчера (сокращение задержек);
\item[2)] повышение вследствие завершения ввода-вывода, причем подходящее значение для повышения зависит от драйвера устройства (таблица~\ref{tab:prority});
\item[3)] повышение, связанные с завершением ожидания;
\item[4)] повышение приоритета владельца блокировки;
\item[5)] повышение при ожидании ресурсов исполняющей системы;
\item[6)] повышение приоритета потоков первого плана после ожидания;
\item[7)] повышение приоритета после пробуждения GUI-потока;
\item[8)] повышения приоритета, связанные с перегруженностью центрального процессора.
\end{enumerate}

\begin{table}[h]
\caption{Рекомендуемые значения повышения приоритета}
\label{tab:prority}
\begin{center}
\begin{tabular}{ | l | l | }
			\hline
			\textbf{Устройство} & \textbf{Приращение} \\ \hline
			Жесткий диск, привод компакт-дисков, & 1 \\ параллельный порт, видеоустройство & \\ \hline
            
            Сеть, почтовый ящик, именованный канал, & 2 \\ последовательный порт & \\ \hline
			Клавиатура, мышь & 6 \\ \hline
			Звуковое устройство & 8 \\ \hline
\end{tabular}
\end{center}
\end{table}

\pagebreak
Категории планирования представлены в таблице~\ref{tab:plan}. 
Функции MMCSS (Multimedia Class Scheduler service) временно повышают приоритет потоков, зарегистрированных с MMCSS до уровня, который соответствует категории планирования. 
Потом их приоритет снижается до уровня, соответствующего категории планирования Exhausted, для того, чтобы другие потоки тоже могли получить ресурс.

\begin{table}[H]
\caption{Категории планирования}
\label{tab:plan}
\begin{tabular}{|>{\raggedleft}p{2.5cm}|>{\raggedleft}p{2.6cm}|>{\raggedright}p{10cm}|}
\hline
\textbf{Категория} & \textbf{Приоритет} & \textbf{Описание}
\tabularnewline
\hline
High & 23--26 & Потоки профессионального аудио (Pro Audio), запущенные с приоритетом выше, чем у других потоков на системе, за исключением критических системных потоков
\tabularnewline
\hline
Medium & 16--22 & Потоки, являющиеся частью приложений первого плана, например, Windows Media Player
\tabularnewline
\hline
Low & 8--15 & Все остальные потоки, не являющиеся частью предыдущих категорий
\tabularnewline
\hline
Exhausted & 1--7 & Потоки, исчерпавшие свою долю времени центрального процессора, выполнение которых продолжится, только если не будут готовы к выполнению другие потоки с более высоким уровнем приоритета
\tabularnewline
\hline
\end{tabular}
\end{table}

\section*{IRQL}

Для обеспечения поддержки мультизадачности системы, когда исполняется код в режиме ядра, Windows использует приоритеты прерываний IRQL (Interrupt Request Level). 
Прерывания обслуживаются в порядке их приоритета. 
При возникновении прерывания с высоким приоритетом процессор сохраняет информацию о состоянии прерванного потока и активизирует сопоставленный с данным прерыванием диспетчер ловушки. 
Последний повышает IRQL и вызывает процедуру обслуживания прерывания (ISR). 
После выполнения ISR диспетчер прерывания понижает IRQL процессора до исходного уровня и загружает сохраненные ранее данные о состоянии машины. 
Прерванный поток возобновляется с той точки, где он был прерван. 
Когда ядро понижает IRQL, могут начать обрабатываться ранее замаскированные прерывания с более низким приоритетом. 
Тогда вышеописанный процесс повторяется ядром для обработки и этих прерываний.

\chapter{Заключение}

Обработчик прерывания от системного таймера для операционных систем семейств Windows и Unix выполняет следующие общие функции:
\begin{enumerate}
\item[1)] декремент счетчиков времени;
\item[2)] декремент кванта;
\item[3)] инициализация отложенных действий, которые относятся к работе планировщика.
\end{enumerate}

Операционные системы семейств Unix и Windows --- это системы разделения времени с динамическими приоритетами и вытеснением.

В операционных системах семейства Unix приоритет пользовательского процесса (процесса в режиме задачи) может динамически пересчитываться, в зависимости от фактора <<любезности>>, \textbf{p\_cpu} (результат последнего измерения использования процессора) и базового приоритета (\textbf{PUSER}). 
Приоритеты ядра --- фиксированные величины.

В операционных системах семейства Windows при создании процесса ему назначается базовый приоритет. 
Относительно базового приоритета процесса потоку назначается относительный приоритет. 
Таким образом, у потока нет своего приоритета. 
Приоритет потока пользовательского процесса может быть динамически пересчитан.

\end{document}
